
\documentclass[a4paper, 12pt]{article} 
\usepackage{amsmath, amssymb, color, graphicx, enumitem}
\usepackage{fullpage} %smaller margins
\usepackage{hyperref} % hyperlinks

%font
%\usepackage[sc]{mathpazo}
%\linespread{1.05}         % Palladio needs more leading (space between lines)
%\usepackage[T1]{fontenc}

%font, libertine
\usepackage{libertine}

%word spacing
\usepackage{microtype}

%all equations get full space
\everymath{\displaystyle}

%useful shortcuts
\def\R{\ensuremath{\mathbb{R}}} %\ensuremath adds math mode, if forgotten
\def\Q{\ensuremath{\mathbb{Q}}}
\def\N{\ensuremath{\mathbb{N}}}
\def\Z{\ensuremath{\mathbb{Z}}}
\def\C{\ensuremath{\mathbb{C}}}

%shorcuts with arguments
\newcommand{\abs}[1]{\left\vert#1\right\vert} %nice absolute values
\newcommand{\bt}[1]{\textbf{#1}} %bold
\newcommand{\eq}[1]{\begin{align*}#1\end{align*}} %aligned equations
\newcommand{\cb}[1]{\centerline{\fbox{#1}}} %centered box
\newcommand{\bp}[1]{\fbox{\parbox{0.8\textwidth}{#1}}} %box paragraph
\newcommand{\norm}[1]{\left\lVert#1\right\rVert} %vector norm
\newcommand{\notimplies}{% does not imply
  \mathrel{{\ooalign{\hidewidth$\not\phantom{=}$\hidewidth\cr$\implies$}}}}
\renewcommand{\eq}[1]{\begin{align*}#1\end{align*}} %aligned equations


%colors
\definecolor{javagreen}{rgb}{0.25,0.5,0.35} %dark green color
\newcommand{\green}[1]{\textcolor{javagreen}{#1}} %command for green
\newcommand{\gray}[1]{\textcolor[gray]{0.5}{#1}} %gray text

%environment
\newcommand{\tab}{\phantom{ssss}}


\title{}
\date{}
%==tips====
%part
    %section, sub, sub
%\begin{enumerate}[resume] %continues counting
\begin{document}
\begin{center}
\section*{Quiz 4}
Fundamentals of Calculus I
\end{center}

Last 4 Digits of Student ID: \underline{\hspace{5cm}} \\

\bt{Explain and justify your thought process.}

\begin{enumerate}
    \item Find the slope of the tangent line to the function 
    $(1+2x + 3x^2)(5x +8x^2-x^3)(2x)$ at $x=1$.
    \vspace{4cm}
\end{enumerate}

\bt{No justification necessary.}

\begin{enumerate}[resume]
    \item When is the product rule useful?
    \vspace{2cm}
    \item State the limit definition of continuity.
    \vspace{2cm}
    \item Find the derivative of $\frac{1}{x}$.
    \vspace{2cm}
\end{enumerate}

\bt{True or False. No justification necessary.}

\begin{enumerate}[resume]
    \item \underline{\hspace{1.5cm}} $\frac{d}{dx} \pi = 0$
    \item \underline{\hspace{1.5cm}} The y-intercept of the secant line is the derivative
    \item \underline{\hspace{1.5cm}} The derivative of a function is also a function
\end{enumerate}

\newpage

\bt{Explain and justify your thought process.}

\begin{enumerate}
    \item Find the slope of the tangent line to the function 
    $(1+2x + 3x^2)(5x +8x^2-x^3)(2x)$ at $x=1$. \\
    \green{
    The slope of the tangent line at $x=1$ is the value of the derivative at 1.
    To find the derivative, we have two options: \\
    * multiply the expression and use the power rule \\
    * group the expression as the product of two functions and use the product rule \\
    For the product rule, we call \\
    $f(x) = 1 + 2x + 3x^2$ \\ and $g(x) = 10x^2 + 16x^3 - 2x^4$.\\
    Then by the power rule, $f'(x) = 2 + 6x$\\ and $g'(x) = 20x + 48x^2 -8x^3.$ \\
    By the product rule the derivative is 
    \eq{ f(x) g'(x) + g(x)f'(x)  
    &= (1+2x+3x^2)(20x+48x^2-8x^3)  \\
    &\ \ \ \ + (10x^2+16x^3-2x^4)(2+6x)
    }
    At x =1, this expression equals 
    $$(1 + 2 + 3)(20 + 48-8) + (10 + 16 -2)(2+6) = 6*60 + 24*8 = 552.$$
    Therefore, the slope of the tanget line at $x=1$ is 552.
    }
\end{enumerate}

\bt{No justification necessary.}

\begin{enumerate}[resume]
    \item When is the product rule useful? \\
    \green{The product rule is used to find the derivative of a function comprised of two functions
    multiplied by one another.
    }
    \item State the limit definition of continuity. \\
    \green{$f(x)$ is continuous at any point $c$ if $\lim_{x \rightarrow c} f(x) = f(c)$
    }
    
    \item Find the derivative of $\frac{1}{x}$. \\
    \green{First rewrite the function as $x^{-1}$ then use the power rule: 
    $-x^{-2}$.
    }
\end{enumerate}

\bt{True or False. No justification necessary.}

\begin{enumerate}[resume]
    \item \green{True} $\frac{d}{dx} \pi = 0$
    \item \green{False} The y-intercept of the secant line is the derivative
    \item \green{True} The derivative of a function is also a function
\end{enumerate}
\end{document}
