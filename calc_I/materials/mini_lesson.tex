\documentclass[a4paper, 12pt]{article} 
\usepackage{amsmath, amssymb, color, graphicx, enumitem}
\usepackage{fullpage} %smaller margins
\usepackage{hyperref} % hyperlinks

%font
%\usepackage[sc]{mathpazo}
%\linespread{1.05}         % Palladio needs more leading (space between lines)
%\usepackage[T1]{fontenc}

%font, libertine
\usepackage{libertine}

%word spacing
\usepackage{microtype}

%all equations get full space
\everymath{\displaystyle}

%useful shortcuts
\def\R{\ensuremath{\mathbb{R}}} %\ensuremath adds math mode, if forgotten
\def\Q{\ensuremath{\mathbb{Q}}}
\def\N{\ensuremath{\mathbb{N}}}
\def\Z{\ensuremath{\mathbb{Z}}}
\def\C{\ensuremath{\mathbb{C}}}

%shorcuts with arguments
\newcommand{\abs}[1]{\left\vert#1\right\vert} %nice absolute values
\newcommand{\bt}[1]{\textbf{#1}} %bold
\newcommand{\eq}[1]{\begin{align*}#1\end{align*}} %aligned equations
\newcommand{\cb}[1]{\centerline{\fbox{#1}}} %centered box
\newcommand{\bp}[1]{\fbox{\parbox{0.8\textwidth}{#1}}} %box paragraph
\newcommand{\norm}[1]{\left\lVert#1\right\rVert} %vector norm
\newcommand{\notimplies}{% does not imply
  \mathrel{{\ooalign{\hidewidth$\not\phantom{=}$\hidewidth\cr$\implies$}}}}
\renewcommand{\eq}[1]{\begin{align*}#1\end{align*}} %aligned equations


%colors
\definecolor{javagreen}{rgb}{0.25,0.5,0.35} %dark green color
\newcommand{\green}[1]{\textcolor{javagreen}{#1}} %command for green
\newcommand{\gray}[1]{\textcolor[gray]{0.5}{#1}} %gray text

%environment
\newcommand{\tab}{\phantom{ssss}}


\title{}
\date{}
%==tips====
%part
    %section, sub, sub
%\begin{enumerate}[resume] %continues counting
\begin{document}
\begin{center}
\section*{(mini-) Lesson}
Fundamentals of Calculus I
\end{center}

Explaining	a	concept	to	others	is	an	excellent	way	to	solidify	your	own	understanding.\\

Your task is to give a clear and engaging min-lesson: 
\begin{itemize}
    \item 8 minute (2 minutes for questions)
    \item one-page lesson plan 
\end{itemize}

All classroom resources including the computer, projector, and board are at your disposal.\\

Assume your audience has not seen the material you are teaching.


\subsection*{Structure}
\begin{enumerate}
    \item \bt{motivate:} Why should we study the topic?
    \item \bt{state and explain:} How does the concept relate to an idea we're familiar with?
    \item \bt{persuade:} Why are the statements about your topic true?
    \item \bt{engage:} Does your interactive activity effectively engage the audience with the topic?
\end{enumerate}

\subsection*{Suggestions}
An important aspect of your task is to determine which ideas are most important and how best to present them.\\

\noindent Examples are a useful way to make a concept clear.\\
Relating a concept to daily experience can make it click. \\
Pictures often capture a concept better than many words. \\


\subsection*{Grading}
The assignment will count as 13\% of your final grade.\\

\noindent Grading is based on the 4 parts outlined in the structure with emphasis on clearly, completely, and engagingly presenting the key ideas of the topic.

\end{document}

