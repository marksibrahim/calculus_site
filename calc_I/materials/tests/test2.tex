

\documentclass[a4paper, 12pt]{article} 
\usepackage{amsmath, amssymb, color, graphicx, enumitem}
\usepackage{fullpage} %smaller margins
\usepackage{hyperref} % hyperlinks

%font
%\usepackage[sc]{mathpazo}
%\linespread{1.05}         % Palladio needs more leading (space between lines)
%\usepackage[T1]{fontenc}

%font, libertine
\usepackage{libertine}

%word spacing
\usepackage{microtype}

%all equations get full space
\everymath{\displaystyle}

%useful shortcuts
\def\R{\ensuremath{\mathbb{R}}} %\ensuremath adds math mode, if forgotten
\def\Q{\ensuremath{\mathbb{Q}}}
\def\N{\ensuremath{\mathbb{N}}}
\def\Z{\ensuremath{\mathbb{Z}}}
\def\C{\ensuremath{\mathbb{C}}}

%shorcuts with arguments
\newcommand{\abs}[1]{\left\vert#1\right\vert} %nice absolute values
\newcommand{\bt}[1]{\textbf{#1}} %bold
\newcommand{\eq}[1]{\begin{align*}#1\end{align*}} %aligned equations
\newcommand{\cb}[1]{\centerline{\fbox{#1}}} %centered box
\newcommand{\bp}[1]{\fbox{\parbox{0.8\textwidth}{#1}}} %box paragraph
\newcommand{\norm}[1]{\left\lVert#1\right\rVert} %vector norm
\newcommand{\notimplies}{% does not imply
  \mathrel{{\ooalign{\hidewidth$\not\phantom{=}$\hidewidth\cr$\implies$}}}}
\renewcommand{\eq}[1]{\begin{align*}#1\end{align*}} %aligned equations

%piecewise function

%\begin{displaymath}
%   f(x) = \left\{
%     \begin{array}{lr}
%       1 & : x \in \mathbb{Q}\\
%       0 & : x \notin \mathbb{Q}
%     \end{array}
%   \right.
%\end{displaymath} 

%colors
\definecolor{javagreen}{rgb}{0.25,0.5,0.35} %dark green color
\definecolor{lightblue}{rgb}{0.149,0.545,0.824} %solarized blue
\definecolor{sred}{rgb}{0.863, 0.196, 0.184} %solarized red

\newcommand{\blue}[1]{{\leavevmode\color{lightblue}{#1}}} %solarized blue 
\newcommand{\green}[1]{{\leavevmode\color{javagreen}{#1}}} %command for green
\newcommand{\red}[1]{{\leavevmode\color{sred}{#1}}} %solarized red
\newcommand{\gray}[1]{{\leavevmode\color[gray]{0.5}{#1}}} %gray text

%environment
\newcommand{\tab}{\phantom{ssss}}

\title{}
\date{}
%==tips====
%part
    %section, sub, sub
%\begin{enumerate}[resume] %continues counting
\begin{document}
\begin{center}
\section*{Test 2}
Fundamentals of Calculus I\\
\end{center}
Name: \\
\vspace{1cm}



Write your answers in the space provided.

\bt{Explain and justify your thought process.}


    \begin{enumerate}
        \item Find the instantaneous rate of change of $f(x) = e^{2x}\ln(xe^x + 1)$ at 
        $x = 2$.
       
        \vspace{8cm}

        \item The world population is modeled by $P(t) = 2560 e^{0.01785* t}$, where
        $t$ is the number of years since $1950$. 
        At what rate is the population growing now in $2015$?
        \vspace{8cm}
    \end{enumerate}

\newpage

\bt{No justification necessary.}

\begin{enumerate}[resume]
        \vspace{1cm}
        \item \underline{\hspace{5cm}} What is the domain of $\frac{3}{x^2 - 36}$?
        \vspace{1cm}
        \item \underline{\hspace{5cm}} Find $\frac{dy}{dt}$ for $y(t) = \sqrt{2t^7 - 5}$.
        \vspace{1cm}
        \item \underline{\hspace{5cm}} What is slope of the tangent line of $f(x) = \pi$ at $x=3$?
        \vspace{1cm}
        \item \underline{\hspace{5cm}} Find the value of $\lim_{x \rightarrow \infty} \frac{2}{3} + \frac{2}{x^2}$.
        \vspace{1cm}
        \item \underline{\hspace{5cm}} Evaluate $\log_2(\log_3(\log_4 64))$
        \vspace{1cm}
        \item \underline{\hspace{5cm}} What are the minimum and maximum values of \\
        \hspace*{5cm} $x^2 + 8x + 35$?
        \vspace{1cm}
        \item \underline{\hspace{5cm}} Find the derivative of $(5x^3 -1)^6$.
        \vspace{1cm}
        \item \underline{\hspace{5cm}} For $f(x) = 1x + 5$ and $g(x) = 3x + 10$, \\
        \hspace*{5cm} find all solutions to $3x = g(f(x))$.
        \vspace{1cm}
        \item Explain the meaning of $\lim_{x \rightarrow 3} f(x) = 5$.
        \vspace{2cm}
        \item State the limit definition of a derivative.
        \vspace{2cm}
\end{enumerate}


\newpage

\section*{Solutions}

Write your answers in the space provided.

\bt{Explain and justify your thought process.}


    \begin{enumerate}
        \item Find the instantaneous rate of change of $f(x) = e^{2x}\ln(xe^x + 1)$ at 
        $x = 2$.

        \green{The instantaneous rate of change is the value of the
        derivative at $x=2$. 
        To find the derivative we first use the product rule:
        $f'(x) = e^{2x} * \frac{d}{dx} ln(xe^x +1) + ln(xe^x + 1) 
        \frac{d}{dx} e^{2x}.$ 

        To find $\frac{d}{dx} ln(xe^x + 1)$ we use the chain rule to obtain
        $$\frac{1}{xe^x +1} * \frac{d}{dx} xe^x
        = \frac{xe^x + e^x}{xe^x +1},$$
        since $\frac{d}{dx} xe^x = xe^x + e^x$ by the product rule.
        
        Therefore, 
        $$f'(x) = \frac{e^{2x} * (x e^x + e^x)}{xe^x + 1} + 2\ln(xe^x + 1) e^{2x}.$$
        At $x=2$, 
        $$f'(2) =  \frac{e^{4} * (2 e^2 + e^2)}{2e^2 + 1} + 2\ln(2e^2 + 1) e^{4}.$$
        }

        \item The world population is modeled by $P(t) = 2560 e^{0.01785* t}$, where
        $t$ is the number of years since $1950$. 
        At what rate is the population growing now in $2015$?

        \green{To find the instantaneous rate of growth in 2015, we need to compute the derivative of $P(t)$.

        By the chain rule we have, 
        $P'(t) = 2560*(0.01785) e^{0.01785*t}$.
        Today in 2015 is equivalent to $t = 65$. 
        Thus the rate at which the population is growing today in 2015 is
        $$P'(65) = 2560*(0.01785)*e^{0.01785*65} = 145.80$$
        (that's in millions)
        }
    \end{enumerate}

\newpage

\bt{No justification necessary.}

\begin{enumerate}[resume]
        \item  What is the domain of $\frac{3}{x^2 - 36}$?\\
        \green{all real number except 6 and -6}

        \vspace{1cm}
        \item Find $\frac{dy}{dt}$ for $y(t) = \sqrt{2t^7 - 5}$.\\
        \green{$\frac{7t^6}{\sqrt{2t^7-5}}$}

        \vspace{1cm}
        \item What is slope of the tangent line of $f(x) = \pi$ at $x=3$?\\
        \green{0}
        \vspace{1cm}
        \item Find the value of $\lim_{x \rightarrow \infty} \frac{2}{3} + \frac{2}{x^2}$.\\
        \green{$\frac{2}{3}$}
        \vspace{1cm}
        \item Evaluate $\log_2(\log_3(\log_4 64))$\\
        \green{0}

        \vspace{1cm}
        \item What are the minimum and maximum values of \\
        \hspace*{5cm} $x^2 + 8x + 35$? \\
        \green{max is infinity; min is 19}

        \vspace{1cm}
        \item Find the derivative of $(5x^3 -1)^6$.\\
        \green{$6(5x^3-1)^5*15x^2$}

        \vspace{1cm}
        \item For $f(x) = 1x + 5$ and $g(x) = 3x + 10$, \\
        \hspace*{5cm} find all solutions to $3x = g(f(x))$.\\
        \green{no solution}
        \vspace{1cm}
        \item Explain the meaning of $\lim_{x \rightarrow 3} f(x) = 5$. \\
        \green{$f(x)$ approaches $5$ when $x$ is close to $3$.}

        \item State the limit definition of a derivative. \\
        \green{$\lim_{h \rightarrow 0} \frac{f(x+h) - f(x)}{h}$}
\end{enumerate}
\end{document}

