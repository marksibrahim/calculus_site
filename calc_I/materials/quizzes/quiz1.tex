

\documentclass[a4paper, 12pt]{article} 
\usepackage{amsmath, amssymb, color, graphicx, enumitem}
\usepackage{fullpage} %smaller margins
\usepackage{hyperref} % hyperlinks

%font
%\usepackage[sc]{mathpazo}
%\linespread{1.05}         % Palladio needs more leading (space between lines)
%\usepackage[T1]{fontenc}

%font, libertine
\usepackage{libertine}

%word spacing
\usepackage{microtype}

%all equations get full space
\everymath{\displaystyle}

%useful shortcuts
\def\R{\ensuremath{\mathbb{R}}} %\ensuremath adds math mode, if forgotten
\def\Q{\ensuremath{\mathbb{Q}}}
\def\N{\ensuremath{\mathbb{N}}}
\def\Z{\ensuremath{\mathbb{Z}}}
\def\C{\ensuremath{\mathbb{C}}}

%shorcuts with arguments
\newcommand{\abs}[1]{\left\vert#1\right\vert} %nice absolute values
\newcommand{\bt}[1]{\textbf{#1}} %bold
\newcommand{\eq}[1]{\begin{align*}#1\end{align*}} %aligned equations
\newcommand{\cb}[1]{\centerline{\fbox{#1}}} %centered box
\newcommand{\bp}[1]{\fbox{\parbox{0.8\textwidth}{#1}}} %box paragraph
\newcommand{\norm}[1]{\left\lVert#1\right\rVert} %vector norm
\newcommand{\notimplies}{% does not imply
  \mathrel{{\ooalign{\hidewidth$\not\phantom{=}$\hidewidth\cr$\implies$}}}}
\renewcommand{\eq}[1]{\begin{align*}#1\end{align*}} %aligned equations


%colors
\definecolor{javagreen}{rgb}{0.25,0.5,0.35} %dark green color
\newcommand{\green}[1]{\textcolor{javagreen}{#1}} %command for green
\newcommand{\gray}[1]{\textcolor[gray]{0.5}{#1}} %gray text

%environment
\newcommand{\tab}{\phantom{ssss}}


\title{}
\date{}
%==tips====
%part
    %section, sub, sub
%\begin{enumerate}[resume] %continues counting
\begin{document}
\begin{center}
\section*{Quiz 1}
Fundamentals of Calculus I
\end{center}

Name: \underline{\hspace{5cm}} \\

\bt{Explain and justify your thought process.}

Write your answers in the space provided.

\begin{enumerate}
    \item What's the equation of the line going through $(2, 5)$ and $(3, 10)$?\\
    \vspace{6cm}

    \item For $f(x) = 1x + 5$ and $g(x) = 3x + 10$, find all solutions to $3x = g(f(x))$. \\
    \vspace{6cm}

    \item Graph $x^2 + 4x + 10$.
    
    \vspace{6cm}

    \item Find all solutions to $x^2 + 4x + 10 = 5$ (hint: see previous question).

    \vspace{6cm}
   For questions 5 and 6, note Apple can build an iphone 6 factory for \$100,000. Each iphone costs \$100 to produce. 
    \item What's the total cost of producing 800 iphones? \\

    \vspace{6cm}
    \item If Apple sells each iphone for \$500, how many iphones does Apple need to sell to earn \$80,000 in profit?
    \vspace{6cm}
\end{enumerate}

\newpage
\centerline{\bt{Solutions}}
\begin{enumerate}
    \item What's the equation of the line going through $(2, 5)$ and $(3, 10)$?\\
    \green{
    First we find the slope. Slope answers the question: how much does y change by when $x$ increases by 1?\\
    When $x$ inceases by 1, $y$ increases from 5 to 10, implying the slope is 5.
    Therefore we have $y = 5x + b \implies 5 = 10 + b \implies b = -5$. Thus the equation of the line is $y = 5x -5$.
    }

    \item For $f(x) = 1x + 5$ and $g(x) = 3x + 10$, find all solutions to $3x = g(f(x))$. \\
    \green{
    No solution, as the lines are parallel after evaluating the function:
    \eq{
    g(f(x)) &= 3(x+5) +10  \\
    & = 3x + 15 + 10  = 3x + 25.}
    }
    \item Graph $x^2 + 4x + 10$.
    \green{
    Complete the square to understand the function:
    \eq{
    x^2 + 4x + 10 = (x+2)^2 + 6
    }
    Therefore the function is $x^2$ shifted to the left by 2 and up by 6.
    }

    \item Find all solutions to $x^2 + 4x + 10 = 5$ (hint: see previous question).
    \green{
    We determined the function is $x^2$ shifted to the left by 2 and up by 6. Thus, the function never achieves a value of 5, meaning there are no solutions.
    }


   For questions 5 and 6, note Apple can build an iphone 6 factory for \$100,000. Each iphone costs \$100 to produce. 
    \item What's the total cost of producing 800 iphones? \\
    \green{if we let $x$ be the number of iphones we have: 
    cost = 100 x + 100,000
    We evaluate our function at an input of 800: 
    cost = 100*800 + 100,000 = 80,000 + 100,000 = 180,000.
    }

    \item If Apple sells each iphone for \$500, how many iphones does Apple need to sell to earn \$80,000 in profit?
    \green{
    If $x$ is the number of iphones sold, 
    \eq{
    \text{profit} &= 500 x - \text{cost } \\
    & = 500x - (100x + 100,000) \\
    & = 400x - 100,000.
    }
    We need to find the input (number of iphones sold) that generates an output (profit) of 80,000: 
    \eq{
    80,000 = 400x - 100,000 \implies 180,000/400 = x = 450.
    }
    Therefore, Apple needs to sell 450 iphones to earn 80,000 in profit. 
    }
\end{enumerate}

\subsection*{Common Mistakes}

\begin{itemize}
    \item \bt{not reading the question carefully} Many students provided only the slope, not the equation of the line in question 1. 
    \item \bt{not realizing the implications of a false statement such as 0 = 25.} For example in problem 2, many students arrived at at an impossible result 0 = 25, but couldn't answer the question. We know 0 never equals 25. Therefore, the statements leading up to 0=25 can't be true either.
    \item \bt{not understanding notation} For example many incorrectly interpreted the notation $f(g(x))$ in question 2.
    This notation means we takes $g(x)$ as an input of the function $f(x)$.
    
    \item \bt{distinguishing a single output from a function.} For example in question 6, many assigned cost or profit a fixed value, instead of realizing each is a function which depends on x (the number of iphones produced/sold).

    \item \bt{repeating procedures from class without understanding the aim of the question.} For example in question 4, many students completed the square or described the domain, without addressing the question. Understanding what "find all solutions" means is a big step towards answering this question. 
\end{itemize}

\end{document}

